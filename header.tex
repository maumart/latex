% Basisformatierungen
\documentclass[fontsize=12pt, oneside, pagesize=auto, ngerman, a4paper, titlepage, openany,
bibliography=totocnumbered, listof=totocnumbered, headings=big, DIV=14, BCOR=5mm,
headinclude=true, footinclude=false, final]{scrbook} %numbers=noenddot,

% Geometry
%\usepackage[inner=2cm, outer=3cm, top=2.5cm, bottom=2.5cm, includeheadfoot]{geometry}

%Symbole
\usepackage{mathabx}

% Subsubsection mit Nummern
\setcounter{secnumdepth}{3}

% Anzahl der Ebenen im Inhaltsverzeichnis
\setcounter{tocdepth}{2}

% Zeilenabstand 1,5em
\usepackage[onehalfspacing]{setspace}

% Absatz Neue Zeile durch Leerzeile im Quellcode
\usepackage{parskip}

% Absatz Neue Zeile
% keine Einrueckung nach rechts
\setlength{\parindent}{0in}

% Bilderumgebung
\usepackage{graphicx}

% Sprache - Neue Rechtschreibung 
\usepackage[ngerman]{babel}

% Umlaute
\usepackage[utf8]{inputenc}

% T1 / Cork encoding
%\usepackage{textcomp}
\usepackage[T1]{fontenc}

% LModern schoene Schrift
%\usepackage{lmodern}
\usepackage[osf,sc]{mathpazo}
\renewcommand{\rmdefault}{pplj}

% Kopf- und Fusszeile
\usepackage{fancyhdr}
\pagestyle{fancy}
\fancyhf{}
%\fancyfoot{}

% Kopfzeile links bzw. innen
%\fancyhead[LO,RE]{\nouppercase{\leftmark}}

% Kopfzeile rechts bzw. aussen
%\fancyhead[RO,LE]{\thepage}

\fancypagestyle{appendix}{%
    %\fancyhead[LO,RE]{Anhang C.~~\nouppercase{\leftmark}}
    %\fancyhead[C]{Anhang C.~~\nouppercase{\leftmark}}
    \fancyhead{}
    %\renewcommand{\headrulewidth}{0.4pt}
}

\fancypagestyle{chapterstyle}{%
    %\fancyhead[LO,RE]{Anhang C.~~\nouppercase{\leftmark}}
    \fancyhead{}
    %\renewcommand{\headrulewidth}{0.4pt}
}

\fancyhead[C]{\nouppercase{\leftmark}}

% Fusszeile Seitenzahl
\fancyfoot[C]{\thepage}

% Linie oben
\renewcommand{\headrulewidth}{0.4pt}

% Linie unten
\renewcommand{\footrulewidth}{0.4pt}

% Kopfzeile auch bei Chapterseiten einfuegen
%\def\chapterpagestyle{fancy}
\def\chapterpagestyle{chapterstyle}

% Schusterjungenverhinderung
\clubpenalty = 10000 % schliesst Schusterjungen aus
\widowpenalty = 10000 % schliesst Hurenkinder aus

% Blindtext
\usepackage{blindtext}

% Für Ausrichtungen von Grafiken und Tabellen
\usepackage{float} 
\restylefloat{table} 

% Package fuer Text neben Bild
\usepackage{wrapfig}

% listings
\usepackage{listings}

% Umlaute fuer Listing umschreiben
\lstset{
  literate={ö}{{\"o}}1
           {ä}{{\"a}}1
           {ü}{{\"u}}1
           {Ü}{{\"U}}1
           {Ö}{{\"O}}1
           {Ä}{{\"A}}1
           {ß}{{\ss}}1
}

% Anhang Name für Listings ändern
\renewcommand{\lstlistlistingname}{Quellcodeverzeichnis}
\renewcommand{\lstlistingname}{Quellcode}

% Listings HTML-Vorlage
\lstdefinestyle{HTML} {
	language=html,
    basicstyle=\scriptsize\ttfamily,
    keywordstyle=\bfseries\ttfamily,
    commentstyle=\color{gray}\ttfamily,
	numbers=left,
	stepnumber=1,
	numbersep=0.75em,
	numberstyle=\small\color{lightgray2}\ttfamily,
	basicstyle=\ttfamily,
	keywordstyle=\color{black},
	commentstyle=\color{black},
	stringstyle=\slshape\color{darkgray2},
	frame=single,
	tabsize=4,	
	backgroundcolor=\color{lightgray2},
	captionpos=b,
	aboveskip=\bigskipamount,
	belowskip=\medskipamount,
    abovecaptionskip=\bigskipamount,
    belowcaptionskip=\bigskipamount,
	showspaces=false,
	breaklines=true,			
	numberblanklines=true,
	showstringspaces= false,
	breakindent = 0pt,
	morekeywords={nav, header, footer, main, aside, article, placeholder, role,
	aria-live, tabindex}
	%keepspaces=true,
	%columns=fullflexible
	%xleftmargin=\lstxframesep,	
	%framerule=0.5pt
	%framesep=0
}

% Bildeunterschrift
%\usepackage[format=hang, justification=centering, singlelinecheck=off]{caption}
\usepackage[center]{caption}
\DeclareCaptionFormat{empty}{} % HACK Unterschriften fuer Tabellen verstecken aber im LoT
% zeigen

% Bildeunterschrift pro Minipage
\usepackage{subcaption}

% Anmerkungen Ende eindeutschen
%\usepackage{endnotes}
%\renewcommand{\notesname}{Anmerkungen}

% Fussnoten nicht bei jedem Chapter resetten
\usepackage{chngcntr} 
\counterwithout{footnote}{chapter}

% Bibtexformat
%\usepackage[style=authortitle-icomp, backend=bibtex]{biblatex}
%\usepackage[style=numeric,firstinits=true, backend=bibtex,hyperref=true]{biblatex}
\usepackage[style=numeric,backend=bibtex,hyperref=true]{biblatex}
 
% Zeilenabstand Anhang
\setlength{\bibitemsep}{1em} 

% Besseres Kerning
%\usepackage{microtype}
\usepackage[activate={true,nocompatibility},final,tracking=true,kerning=true,spacing=true,
factor=1100,stretch=10,shrink=10]{microtype}

% Weissraum Auslassungspunkte
\usepackage{ellipsis}

% Eurosymbol 
\usepackage{eurosym}

% Graphiken
\usepackage{color}
\usepackage{pstricks} 
\usepackage{transparent}

% Todonotes
\usepackage{todonotes}
\setlength{\marginparwidth}{2.2cm}

%Multicol
\usepackage{multicol}

%Booktabs Tables
\usepackage{tabularx}
\usepackage{booktabs}
\setlength{\defaultaddspace}{10pt}

% Abkuerzungshelper
\usepackage{xspace}
\newcommand{\zB}{\mbox{z.\,B.}\xspace}
\newcommand{\html}{\mbox{HTML\,5}\xspace}
\newcommand{\bspw}{beispielsweise\xspace}
\newcommand{\sAbb}{siehe Abbildungen\xspace}
\newcommand{\bFh}{Barrierefreiheit\xspace}

%Acronyme
%\usepackage[printonlyused]{acronym}
\usepackage{acronym}

%bar Chart
\usepackage{bchart}

%Pie Chart
\usepackage{calc}
\usepackage{ifthen}
\usepackage{tikz}
\newcommand{\slice}[4]{
  \pgfmathparse{0.5*#1+0.5*#2}
  \let\midangle\pgfmathresult

  % slice
  \draw[thick,draw=black,fill=lightgray2] (0,0) -- (#1:1) arc (#1:#2:1) -- cycle;

  % outer label
  \node[label=\midangle:#4] at (\midangle:1) {};

  % inner label
  \pgfmathparse{min((#2-#1-10)/110*(-0.3),0)}
  \let\temp\pgfmathresult
  \pgfmathparse{max(\temp,-0.5) + 0.8}
  \let\innerpos\pgfmathresult
  \node at (\midangle:\innerpos) {#3};
}

% URL schoener formatieren
%\usepackage[hyphens]{url} 
\PassOptionsToPackage{hyphens}{url}
\usepackage[ 
   colorlinks=true, 
   breaklinks=true,        
   linkcolor=black,  
   filecolor=black,  
   citecolor=black, 
   pdfborder=black,  
   urlcolor=black,
   plainpages=false
]{hyperref}

%\usepackage{breakurl}